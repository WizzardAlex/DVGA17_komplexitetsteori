\section{Del 2: Tornen i Hanoi}
\paragraph{Inledning}    
Vi har tre pinnar med givna respektive namn A, B, C, samt notationen N för antal ringar.
\newline
Regler:
\begin{enumerate}
\item Flytta en ring åt gången.
\item Lägg aldrig en större ring ovanpå en mindre.
\end{enumerate}

\newline
Given rekursiv algoritm:
\begin{enumerate}
	\item Flytta temporärt högen med N − 1 ringar från pinne A till pinne B.
	\item Flytta den största ringen (den enda ringen kvar på pinne A) till pinne C.
	\item Flytta N − 1 ringar från pinne B till pinne C.
\end{enumerate}

\paragraph{Omskrivning}

görs för att komma fram till en tidskomplexitet av den rekursiva algoritmen till
en rekursiv psudokod/funktion. Som kan beskrivas, funktionen H(antal ringar, som flyttas från
pinne x, till pinne y)

\textbf{Tidsfunktion} 
\vspace{5mm} 
\\ H( N, A, C) 
\\	H( N - 1, A, B)	\hspace{1cm}    // kallar på sig självt
\\	flytt(A,C)     	 \hspace{1cm}   // flyttar ringen ett steg	
\\	H( N - 1, B, C)	 \hspace{1cm}   // kallar på sig självt 
\vspace{5mm}
\\ Med tiden i åtanke kan flytt(A,C) ses som en konstant ökning i tidskomplexiteten.
\newline Därav kan ovan skrivas om till tidsfunktionen 
\begin{align*}
$$
    T(N) &= T(N-1) + 1 + T(N-1) \\ 
         &= 2T(N-1) + 1 
$$
\end{align*}
\newline
Med hjälp av inducering fås.
\begin{align*}
  $$  
    T(N) &= 2[2T(N-1-1)+1] + 1	  \\  &= 2^2 T(N-2) + 3
\\	&= 2^2[2T(N-3)+1] + 3	  \\  &= 2^3 T(N-3) + 7 
    $$
\end{align*}
Då k är antal induceringar för funktionen $ T(N) = 2^k T(N-k) + 2^k - 1 $
\\ När $ k \rightarrow N$ så fås tidsfunktionen 
\begin{align*}
  $$   T(N) &= 2^N T(N-N) + 2^N - 1 \\
    &= 2^N T(0) + 2^N -1 \\
    &= 0 + 2^N - 1 = 2^N - 1 
  $$ 
\end{align*}
\vspace{5mm}
Detta ger Tornen i Hanoi tidskomplexiteten $\mathcal{O}(2^n)$ då konstanten kan bortses.

